% This is a LaTeX file
% by David W. Hogg

\documentstyle[12pt,fullpage,psfig]{article}

% do latin stuff
  \newcommand{\latin}[1]{\textit{#1}}
  \newcommand{\etal}{\latin{et~al}}
  \newcommand{\etc}{\latin{etc.}}
  \newcommand{\ie}{\latin{i.e.}}

\begin{document}
\title{Magnitude systems\footnote{
Originally published as Appendix A in David W. Hogg, 1998, \textit{On
the evolution of field galaxies}, PhD Thesis, California Institute of
Technology.}}
\author{David W. Hogg}
\date{1997 November}
\maketitle

\section{Vega-relative magnitudes}

The system of {\em apparent magnitudes,\/} or simply {\em
magnitudes,\/} is a logarithmic flux scale, defined such that the
standard star Vega has magnitude zero in all bands.  If Vega has flux
$S_0$ in some band, an object with flux $S$ is assigned the magnitude
$m=-2.5\log(S/S_0)$.  Brighter objects have smaller magnitudes.  The
magnitude definitions and absolute calibrations (\ie, zeropoints or
Vega fluxes) used in this dissertation are given in
Table~\ref{tab:magtable}.

Magnitudes are designed for relative measurements, which, given the
poorly understood, constantly changing properties of the atmosphere,
are the only robust and precise measurements possible with groundbased
telescopes.  While it is straightforward (if, perhaps, not easy) to
measure the relative fluxes of two astronomical objects at an accuracy
of $10^{-3}$, it is very difficult to measure an absolute flux to
better than about 5~percent.  Any such measurement requires accounting
for absorption by the Earth's atmosphere (which is time- and
airmass-dependent) and a precise ``laboratory'' calibration of the
instrument plus telescope system used to detect the light.  For this
reason, the absolute calibrations given in Table~\ref{tab:magtable}
should be taken to be approximate.  An additional reason for caution
is that the data in Table~\ref{tab:magtable} are largely from
secondary sources.

As an added bonus, the calibration chart used by Neugebauer (private
communication) is given in Table~\ref{tab:gxnmagtable}.  The
differences between Tables~\ref{tab:magtable} and
\ref{tab:gxnmagtable} are small, even though they are
based on at least partially independent information.


% ----------------------------------------------------------------------------
\section{A note on absolute calibration}
\label{sec:hayes}

Actually, the problem with absolute flux calibration of photometry
goes deeper than the simple fact, mentioned in the previous Section,
that it is difficult to measure.  The flux calibrations given in
Tables~\ref{tab:magtable} and \ref{tab:gxnmagtable} in fact contain an
arbitrary convention:

The absolute calibration of the magnitude systems (the $S_{\nu}^{(0)}$
values in Table~\ref{tab:magtable}) are given in terms of a flux
density (flux per unit log frequency), which is defined only at a
single point in the spectrum, while any photometric bandpass in fact
has a finite width, and probably a non-trivial profile.  In practice,
for each bandpass, an effective wavelength $\lambda_{\rm eff}$ is
chosen, at which the zeropoint is correct.  How does one compute
$\lambda_{\rm eff}$ for a given bandpass?  Should one take the mean of
the transmission function?  If so, in wavelength space or frequency
space or log-frequency space?  Different choices give different
results.  Once a choice has been made, the calibration will only be
exactly correct (in the sense that the true $S_{\lambda}$ of the
source at $\lambda_{\rm eff}$ equals the photometrically-inferred
value) for one particular spectral shape; all other spectral shapes
will require a color-correction.  For example, a very red object might
have all its flux in a bandpass coming from the very reddest ten
percent of the bandpass, or even from a part of a long wavelength tail
caused by a ``red leak'' in the filter.  For another, a source might
be emitting all its flux in a single narrow line which does not happen
to lie exactly at $\lambda_{\rm eff}$.  In practice, with visual and
near-infrared bandpasses, these color-corrections are usually small
because source spectra tend to be well-behaved and the usual
bandpasses tend to be relatively narrow.  However, this is a
fundamental limitation to the absolute calibration of broad-band
photometric bandpasses, and another reason to treat all absolute
calibrations with caution.

Of course this abiguity can be seen as a blessing.  Given a bandpass,
absolute calibration merely requires the (arbitrary?) choice of a
$\lambda_{\rm eff}$ and someone else's painstakingly measured flux
density $\log\lambda\,S_{\lambda}$ of Vega at that wavelength.  The
best absolute calibration of Vega in the visual as of this
dissertation is due to Hayes (1985) and shown in
Figure~\ref{fig:hayes}; it is what was used to calibrate the
photographic and Gunn bandpasses in Table~\ref{tab:magtable}.


% ----------------------------------------------------------------------------
\section{Absolute magnitudes}

The absolute magnitude $M$ is a measure of luminosity.  It is the apparent
magnitude the object would have if it were at 10~pc distance, so
\begin{equation}
M = m - 5\,\log\left(\frac{D}{\rm 10~pc}\right)
\end{equation}
where $D$ is the luminosity distance to the object (ignoring
k-correction).  Because Vega is rougly at 10~pc distance, this system
is also more-or-less Vega-relative.  Because all observable
extragalactic objects are far more luminous than Vega, their absolute
magnitudes will all be negative.

To convert the flux calibrations given in Table~\ref{tab:magtable}
into luminosity calibrations, \ie, log luminosities
$\log(\nu\,L_{\nu}^{(0)})$, in Watts, of an absolute magnitude $M=0$
object, simply add the logarithm of $4\pi(10~{\rm pc})^2$ in meters,
or $36.08$.


% ----------------------------------------------------------------------------
\section{``AB'' magnitudes}

The ``AB'' magnitude system was designed to have absolute calibrations
which are the same in $f_{\nu}$ for all bands, instead of having
calibrations equal to the flux of Vega in each band.  By definition,
AB and Vega-relative magnitudes are equal in the $V$ band.  Note that
construction of the AB system requires absolute calibration of the
magnitude scales, so correct AB magnitudes cannot known any better
than the flux calibrations, despite the fact that Vega-relative
magnitudes can be known to arbitrary accuracy.  This fact alone
reccomends against using AB magnitudes except in special
circumstances.  Furthermore, as discussed above, there is a
conventional or arbitrary component to the flux calibration!

To compute an AB calibration in the units employed in
Table~\ref{tab:magtable}, take the $V$-band calibration, add $\log\nu$
for the band in question, and subtract $\log\nu$ for the $V$ band.
So, for instance, the $K_{AB}$ calibration is $\log(\nu
S_{\nu}^{(0)})=-8.31$ (in ${\rm W\,m^{-2}}$).


% ----------------------------------------------------------------------------
\section{Transformations between magnitudes}

Frequently a transformation between different magnitudes is required,
for example when the $V$ and $I$-band magnitudes of a source are known
and the $R$-band needs to be predicted.  In this case some assumption
needs to be made about the spectral energy distribution of the source.
For extragalactic work the best assumption is that the distribution is
a power law, $\nu\,f_{\nu}\propto\nu^n$.  Then, to a reasonable
approximation, the $R$-band flux can be found by interpolating between
the $V$ and $I$-band fluxes (found using the absolute calibrations of
the $V$ and $I$ bands) on a log-log plot.  The $R$-band absolute
calibration is then applied to get an $R$-band magnitude.  This
procedure generalizes to the following rule:

To get the best estimate of a magnitude $m_C$ in band $C$ given
magnitudes $m_A$ and $m_B$ in bands $A$ and $B$, use
\begin{equation}
m_C=a\,m_A+b\,m_B+c
\end{equation}
where
\begin{eqnarray}
a & = & \frac{\log\nu_C-\log\nu_B}{\log\nu_A-\log\nu_B} \\
b & = & \frac{\log\nu_A-\log\nu_C}{\log\nu_A-\log\nu_B} \\
c & = & -2.5\,(a\,Z_A+b\,Z_B-Z_C)
\end{eqnarray}
where the $\log\nu_i$ are the effective wavelengths of each band $i$
and the $Z_i$ are the absolute calibrations $\log[\nu\,f_{\nu}^{(0)}]$.


% ----------------------------------------------------------------------------
\section*{Acknowledgements}
\addcontentsline{toc}{section}{Acknowledgements}

Thanks go to Gerry Neugebauer for providing zeropoints and for
pointing out the fundamental ambiguity in absolute calibration.


% ----------------------------------------------------------------------------
\section*{References}
\addcontentsline{toc}{section}{References}
\begin{list}{}{
  \item \rightmargin=0in
  \leftmargin=0.25in
  \topsep=0ex
  \partopsep=0pt
  \itemsep=1ex
  \parsep=0pt
  \itemindent=-1.0\leftmargin
  \listparindent=\leftmargin
}
\item
Fukugita M., Shimasaku K. \& Ichikawa T., 1995, Galaxy colors in
various photometric band systems, PASP 107 945
\item
Hayes D. S., 1985, Stellar absolute fluxes and energy distributions
from 0.32 to 4.0 $\mu$m, in Hayes D. S. \etal\, eds., {\em Proc IAU
111: Calibration of Fundamental Stellar Quantities,} Kluwer,
Dordrecht, 225
\item
Holtzman J. A., Burrows C. J., Casertano S., Hester J. J., Trauger J.
T., Watson A. M. \& Worthey G., 1995, The photometric performance and
calibration of WFPC2, PASP 107 1065
\item
Steidel C. C. \& Hamilton D., 1993, Deep imaging of high redshift QSO
fields below the lyman limit II:\ Number counts and colors of field
galaxies, AJ 105 2017
\item
Zombeck M. V., 1990, {\it Handbook of Space Astronomy and
Astrophysics,\/} Cambridge University, Cambridge
\end{list}


% ----------------------------------------------------------------------------
\clearpage
\begin{table}[p]
\begin{center}
\begin{tabular}{cllll}
band & $\lambda_{\rm eff}$ & $\Delta\lambda$ & $\log\nu$ & $\log\nu\,S_{\nu}^{(0)}$ \\
     & ($\mu{\rm m}$) & ($\mu{\rm m}$) & (Hz)  & (${\rm W\,m^{-2}}$) \\[0.5ex]
\hline
$F300W$  & 0.29  & 0.073 & 15.01 & $-7.98$ \\
$U_n$    & 0.36  &       & 14.92 & $-7.89$ \\
$U$      & 0.365 & 0.068 & 14.91 & $-7.81$ \\
$B$      & 0.44  & 0.098 & 14.83 & $-7.54$ \\
$F450W$  & 0.45  & 0.096 & 14.82 & $-7.54$ \\
$B_J$    & 0.46  & 0.15  & 14.81 & $-7.56$ \\
$G$      & 0.48  &       & 14.80 & $-7.61$ \\
$V$      & 0.55  & 0.089 & 14.74 & $-7.70$ \\
$F606W$  & 0.59  & 0.15  & 14.71 & $-7.77$ \\
$r$      & 0.65  & 0.089 & 14.66 & $-7.86$ \\
$\cal R$ & 0.69  &       & 14.64 & $-7.92$ \\
$R$      & 0.70  & 0.22  & 14.63 & $-7.91$ \\
$F814W$  & 0.79  & 0.15  & 14.58 & $-8.03$ \\
$I$      & 0.90  & 0.24  & 14.52 & $-8.12$ \\
$J$      & 1.25  & 0.3   & 14.38 & $-8.40$ \\
$H$      & 1.65  & 0.4   & 14.26 & $-8.71$ \\
$K_s$    & 2.15  & 0.3   & 14.14 & $-9.01$ \\
$K$      & 2.2   & 0.4   & 14.13 & $-9.04$ \\
$L$      & 3.6   & 1.2   & 13.92 & $-9.65$ \\
\end{tabular}
\end{center}
\caption[Vega-relative magnitude wavelengths, frequencies and absolute calibrations.]{
The Vega-relative magnitude wavelengths, frequencies and absolute
calibrations used in this dissertation.  The full width (\ie, not
half width) of the bandpass is symbolized $\Delta\lambda$.
Frequencies are given in Hz and absolute calibrations are given in
flux per unit ln wavelength $\nu\,S_{\nu}=\lambda\,S_{\lambda}$.  Data
for the custom HST bandpass magnitudes ($F300W$ \etc) are from
Holtzman \etal\ (1995).  Data for $U_n$, $G$ and $\cal R$ are from
Steidel \& Hamilton (1993; where the Vega-relative calibration on
their AB system is given with the wrong sign---when corrected it
provides the above calibrations).  Filter information for $B_J$ and
$r$ are from Fukugita \etal\ (1995), while the calibrations are my own
calculation (using data in Hayes 1985 and the method described in
Section~\ref{sec:hayes}).  Filter information for $K_s$ come from
Neugebauer (private communication) and the calibration from assuming
that Vega is a hot blackbody in the region of the $K$ band.  Data for
the remaining Johnson magnitudes are from a (very) secondary source
(Zombeck 1990).  No calibration should be treated with any more
respect than it deserves (see text).}
\label{tab:magtable}
\end{table}

\begin{table}[p]
\begin{verbatim}
                                                  11 Dec 81, GXN
                                                  updated 9 Mar 97

                      ABSOLUTE CALIBRATION

                                      Zero magnitude
Band    lam     log[nu]         flam    fnu    log[fnu]  m0   notes
        eff                     1E-11   Jy
        mu      log[Hz]        W/m2 mu        lg[W/m2Hz] mag
U       0.36    14.92           3980    1720    -22.76  15.59   1
B       0.43    14.84           7285    4490    -22.35  16.63   1
V       0.548   14.74           3526    3530    -22.45  16.37   2
R       0.7     14.63           1702    2780    -22.56  16.11   1
I       0.9     14.52           830     2240    -22.65  15.88   1
J       1.25    14.38           303     1578    -22.80  15.50   2
H       1.65    14.26           115     1041    -22.98  15.04   2
K       2.2     14.13           40      646     -23.19  14.53   2
L       3.5     13.93           6.8     278     -23.56  13.61   2
L'      3.7     13.91           5.5     251     -23.60  13.50   2
M       4.8     13.80           2.02    155     -23.81  12.98   2
N       10.1    13.47           0.109   37      -24.43  11.42   3
O       20.2    13.17           0.0074  10      -25.00  10.00   3

 AB == -2.5 * log f nu - 56.13
 log[f nu(mJy)] == (m0 - m)*0.4

 Vega defined as 0 Mag for V and 1.25 <= lambda <= 4.8 mu.

 1) Hayes 1979, Dudley Obs. RN. 14, 297.
 2) Vega flux from Oke and Shild 1970, Ap.J., 161, 1015.
    Kurucz, Peytreman and Avrett model 1972.
 3) Becklin 1972 calibration.
\end{verbatim}
\caption[Neugebauer's Vega-relative magnitude wavelengths, frequencies
and zeropoints.]{
Vega-relative magnitude wavelengths, frequencies and zeropoints
according to Neugebauer (private communication).}
\label{tab:gxnmagtable}
\end{table}

\begin{figure}
\psfig{figure=hayespretty.eps,width=\textwidth}
\caption[The Hayes (1985) calibrated spectrum of Vega.]{
The Hayes (1985) calibrated spectrum of Vega, plotted in
$\log\lambda\,S_{\lambda}=\log\nu\,S_{\nu}$, in SI units.  As
described in the text, Vega-relative broad-band photometry can be
calibrated with this spectrum.}
\label{fig:hayes}
\end{figure}

\end{document}
