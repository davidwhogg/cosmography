% this is a plain TeX file

% From - Fri Aug  3 16:00:32 2001
% X-UIDL: 2087-991169591
% X-Mozilla-Status: 0013
% X-Mozilla-Status2: 00000000
% Return-Path: <pjep@PUPGG.PRINCETON.EDU>
% Received: from e1g1.home.nyu.edu ([192.168.78.21]) by
%           homemail.nyu.edu (Netscape Messaging Server 4.15) with ESMTP id
%           GHI73K00.5EZ for <dwh2@nyu.edu>; Fri, 3 Aug 2001 13:54:56 -0400 
% Received: from PUPGG.PRINCETON.EDU (localhost [127.0.0.1])
% 	by e1g1.home.nyu.edu (8.10.1/8.10.1) with ESMTP id f73Hsts23944
% 	for <david.hogg@nyu.edu>; Fri, 3 Aug 2001 13:54:55 -0400 (EDT)
% Received: by PUPGG.PRINCETON.EDU (MX V5.2 Vn8h) id 73;
%           Fri, 3 Aug 2001 13:56:21 -0400
% Date: Fri, 3 Aug 2001 13:56:33 -0400
% From: pjep@PUPGG.PRINCETON.EDU
% To: david.hogg@nyu.edu
% CC: pjep@PUPGG.PRINCETON.EDU
% Message-ID: <009FFFAC.917B3D00.73@PUPGG.PRINCETON.EDU>
% Subject: RE: Review
% MIME-Version: 1.0
% Content-Transfer-Encoding: 7bit
% Content-Type: text/plain; charset="iso-8859-1"

\magnification\magstep1
\baselineskip 5truemm plus 0.1truemm
\parskip 3truemm
\parindent 0pt

Hi David

This is submitted as a plain TEX file, that may generate extra
credit for neatness. 

The Appendix certainly has errors/typos, and a convention I've
not seen before. I'd be glad to instructed about the latter if
you see something I don't.  

A. First equation:

In the convention I'm used to, an object that has physical length
$l$ transverse to the line of sight, and is observed at redshift
$z$, subtends angle 
$$
\theta = l/[a(z)x(z)],\eqno(1)
$$
where $x(z)$ is the angular size distance and $a=a_o/(1 + z)$ is
the expansion parameter. In the first
equation in the appendix, the `comoving radius' is 
$$
r(z) = \sqrt{1-\Omega}\, H_oa_ox(z),\eqno(2)
$$
where the subscript means present value and $c=1$. 

Do you agree? Can you suggest the role/meaning of the prefactor
$\sqrt{1-\Omega}$? 

B. Second equation:

The energy flux density, erg sec$^{-1}$ cm$^{-2}$, received from
an object with bolometric luminosity $L$ at redshift $z$ and
angular size distance $x(z)$ is
$$
f = {L\over 4\pi a_o^2x(z)^2(1+z)^2} =
{LH_o^2(1-\Omega )\over 4\pi r(z)^2(1+z)^2},\eqno(3)
$$
in terms of the authour's $r(z)$. In terms of the `luminosity
distance' $D$ defined in the second equation, this is
$$
f = {L(1 - \Omega )^2\over 4\pi D^2}.\eqno(4)
$$
He can't mean to define the luminosity distance this way; there
has to be a typo at best. Agreed? 

C. Third equation:

I don't know what a `comoving volume' is, and am not sure I want
to. 

D. Equation for $H_ot$:

The familiar solution in parametric form in a open model involves
hyperbolic sines and cosines, so the parameter can be eliminated
by using the quadratical equation. I've not seen the calculation.
If correct this could be handy, though using the parametric
equation isn't all that hard. 

E. Last equation on page 757:

This repeats the mysterious convention $r\propto\sqrt{1-\Omega}x$;
otherwise the equation seems OK. 

F. Top equation on page 758:

Why didn't he number these equations? Why has the mystery factor
$\sqrt{1-\Omega}$ at last disappeared? 

G. The last equation:

This one is OK apart from sloppy typesetting: the argument of the
inverse sinh is the numerator only. 

Cheers, Jim 

\bye
 
\end
