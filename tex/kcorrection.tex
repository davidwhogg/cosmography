% to-do
% -----
% - Make separate g_Q and g_R?  Hogg thinks probably yes but hates the
%   requisite diversification of symbols.
% - Give some example cases and figures?  Hogg torn, but thinks probably
%   yes we ought to give two simple example calculations.
% - Match equations in Oke and Sandage exactly?  Hogg thinks no, but we
%   ought to state explicitly how to convert from our terminology to
%   theirs and back again.

\documentclass[preprint]{aastex}

% notation definitions
\newcommand{\kcorrection}{$K$~correction}
\newcommand{\kcorrections}{{\kcorrection}s}
\newcommand{\nuobs}{\nu_o}
\newcommand{\nuemit}{\nu_e}
\newcommand{\lambdaobs}{\lambda_o}
\newcommand{\lambdaemit}{\lambda_e}

\begin{document}

\title{\scalebox{1.5}{The \kcorrection}}
\author{
  David W. Hogg\altaffilmark{1,2},
  Michael Blanton\altaffilmark{1},
  Daniel J. Eisenstein\altaffilmark{3},
  and Ivan K. Baldry\altaffilmark{4}\\
  \textsl{fourth draft---2002 August 15}
}
\altaffiltext{1}{
  Center for Cosmology and Particle Physics,
  Department of Physics,
  New York University
}
\altaffiltext{2}{
  \texttt{david.hogg@nyu.edu}
}
\altaffiltext{3}{
  Steward Observatory,
  University of Arizona
}
\altaffiltext{4}{
  Department of Physics and Astronomy,
  The Johns Hopkins University
}

\begin{abstract}
The \kcorrection\ is used to relate the emitted- or rest-frame
absolute magnitude of a source in one broad photometric bandpass to
the observed-frame apparent magnitude of the same source in another
broad bandpass.  This short paper provides definitions, equations and
pedagogical discussion related to the \kcorrection.
\end{abstract}

\section{Introduction}

The expansion of the universe provides astronomers with the benefit
that recession velocities can be translated into radial distances
\citep[eg,][and references therein]{hogg99cosm}.  It also presents the
challenge that sources observed at different redshifts are sampled, by
any particular instrument, at different rest-frame frequencies.  The
transformations between observed and rest-frame broad-band photometric
measurements are known as ``\kcorrections'' \citep*{humason56a,
oke68a}.

Here we define the \kcorrection\ and give equations for its
calculation, with the goals of explanation, clarification, and
standardization of terminology.

In what follows, we consider a source observed at redshift $z$,
meaning that a photon observed to have frequency $\nuobs$ was emitted
by the source at frequency $\nuemit$ with
\begin{equation}
\nuemit = \nuobs\,[1+z] \;\;\;.
\end{equation}
The apparent flux of the source is imagined to be measured through an
observed-frame bandpass $R$ and the intrinsic luminosity is imagined
to be measured through a emitted-frame bandpass $Q$.  The
\kcorrection\ is used to relate these two quantites.

Technically, the \kcorrection\ described here includes a slight
generalization from the original conception: The observed and
emitted-frame bandpasses are permitted to have arbitrarily different
shapes and positions in frequency space.

\section{Equations}

Consider a source observed to have apparent magnitude $m_R$ when
observed through photometric bandpass $R$, for which one wishes to
know its absolute magnitude $M_Q$ in emitted-frame bandpass $Q$.  The
\kcorrection\ $K_{QR}(z)$ for this source is \emph{defined} by
\begin{equation}
\label{eq:definition}
m_R = M_Q + DM + K_{QR}(z) \;\;\;,
\end{equation}
where $DM$ is the distance modulus, defined by
\begin{equation}
DM = 5\,\log_{10}\left[\frac{D_L}{10~\mathrm{pc}}\right] \;\;\;,
\end{equation}
where $D_L$ is the luminosity distance \citep[eg,][]{hogg99cosm} and
$1~\mathrm{pc}= 3.086\times 10^{16}~\mathrm{m}$.

The apparent magnitude $m_R$ of the source is related to its spectral
density of flux $f_{\nu}(\nu)$ (power per unit area per unit
frequency) by
\begin{equation}
m_R = -2.5\,\log_{10}\left[
  \frac{\displaystyle
          \int\frac{\mathrm{d}\nuobs}{\nuobs}\,f_{\nu}(\nuobs)\,R(\nuobs)}
       {\displaystyle
          \int\frac{\mathrm{d}\nuobs}{\nuobs}\,g_{\nu}(\nuobs)\,R(\nuobs)}
\right] \;\;\;,
\end{equation}
where the integrals are over the observed frequencies $\nuobs$;
$g_{\nu}(\nu)$ is the spectral density of flux for the ``standard
source'', which, for Vega-relative magnitudes, is Vega (or a weighted
sum of a certain set of A0 stars), and, for AB magnitudes, is a
hypothetical constant source with $g_{\nu}(\nu)=3631~\mathrm{Jy}$
(where $1~\mathrm{Jy}= 10^{-26}~\mathrm{W\,m^{-2}\,Hz^{-1}}=
10^{-23}~\mathrm{erg\,cm^{-2}\,s^{-1}\,Hz^{-1}}$) at all frequencies
$\nu$; and $R(\nu)$ describes the bandpass, as follows:

The value of $R(\nu)$ at each freqency $\nu$ is the mean contribution
of a photon of frequency $\nu$ to the output signal from the detector.
If the detector is a photon counter, like a CCD, then $R(\nu)$ is just
the probability that a photon of frequency $\nuobs$ gets counted.  If
the detector is a bolometer or calorimeter, then $R(\nu)$ is the
energy deposition $h\,\nu$ per photon times the fraction of photons of
energy $\nu$ that get absorbed into the detector.  If $R(\nu)$ has
been properly computed, there is no need to write different integrals
for photon counters and bolometers.  Note that there is an implicit
assumption here that detector nonlinearities have been corrected.

The absolute magnitude $M_Q$ can be defined to be the apparent
magnitude that the source \emph{would} have if it were
$10~\mathrm{pc}$ away, at rest (ie, not redshifted), and compact.  It
is related to the spectral density of the luminosity $L_{\nu}(\nu)$
(power per unit frequency) of the source by
\begin{equation}
M_Q = -2.5\,\log_{10}\left[
  \frac{\displaystyle
          \int\frac{\mathrm{d}\nuemit}{\nuemit}\,
              \frac{L_{\nu}(\nuemit)}{4\pi\,(10~\mathrm{pc})^2}\,Q(\nuemit)}
       {\displaystyle
          \int\frac{\mathrm{d}\nuemit}{\nuemit}\,g_{\nu}(\nuemit)\,Q(\nuemit)}
\right] \;\;\;,
\end{equation}
where the integrals are over emitted (ie, rest-frame) frequencies
$\nuemit$, $D_L$ is the luminosity distance, and $Q(\nu)$ is the
equivalent of $R(\nu)$ but for the bandpass $Q$.  As mentioned above,
this does not require $Q=R$, so this is, technically, a generalization
of the \kcorrection.

If the source is at redshift $z$, then its luminosity is related to
its flux by
\begin{equation}
\label{eq:luminosity}
L_{\nu}(\nuemit) = \frac{4\pi\,D_L^2}{1+z}\,f_{\nu}(\nuobs) \;\;\;,
\end{equation}
\begin{equation}
\nuemit = \nuobs\,[1+z] \;\;\;.
\end{equation}
The factor of $(1+z)$ in the luminosity expression
(\ref{eq:luminosity}) accounts for the fact that the flux and
luminosity are not bolometric but densities per unit freqency.  The
factor would appear in the numerator if the expression related flux
and luminosity densities per unit wavelength.

Equation (\ref{eq:definition}) holds if the \kcorrection\ $K_{QR}(z)$
is
\begin{equation}
\label{eq:kcorrection}
K_{QR}(z) = -2.5\,\log_{10}\left[(1+z)\,
  \frac{\displaystyle
          \int\frac{\mathrm{d}\nuobs}{\nuobs}\,f_{\nu}(\nuobs)\,R(\nuobs)\,
          \int\frac{\mathrm{d}\nuemit}{\nuemit}\,g_{\nu}(\nuemit)\,Q(\nuemit)}
       {\displaystyle
          \int\frac{\mathrm{d}\nuobs}{\nuobs}\,g_{\nu}(\nuobs)\,R(\nuobs)\,
          \int\frac{\mathrm{d}\nuemit}{\nuemit}\,
            f_{\nu}\left(\frac{\nuemit}{1+z}\right)\,Q(\nuemit)}
\right] \;\;\;,
\end{equation}

Equation (\ref{eq:kcorrection}) can be taken to be an operational
definition, therefore, of the \kcorrection, from observations through
bandpass $R$ of a source whose absolute magnitude $M_Q$ through
bandpass $Q$ is desired.  In principle, if the $R$ and $Q$ have
different zeropoint definitions (eg, if $R$ is Vega-relative and $Q$
is AB), the $g_{\nu}(\nuemit)$ in the numerator can be a different
function from the $g_{\nu}(\nuobs)$ in the denominator; in practice
this situation is rare.

In the above, all calculations were performed in frequency units.  In
wavelength units, the spectral density of flux $f_{\nu}(\nu)$ per unit
frequency is replaced with the spectral density of flux
$f_{\lambda}(\lambda)$ per unit wavelength using
\begin{equation}
\nu\,f_{\nu}(\nu) = \lambda\,f_{\lambda}(\lambda) \;\;\;,
\end{equation}
\begin{equation}
\lambda\,\nu = c \;\;\;,
\end{equation}
where $c$ is the speed of light.  The \kcorrection\ becomes
\begin{equation}
K_{QR}(z) = -2.5\,\log_{10}\left[\frac{1}{(1+z)}\,
  \frac{\displaystyle
  \int\mathrm{d}\lambdaobs\,\lambdaobs\,f_{\lambda}(\lambdaobs)\,R(\lambdaobs)\,
    \int\mathrm{d}\lambdaemit\,\lambdaemit\,
    g_{\lambda}(\lambdaemit)\,     Q(\lambdaemit)}
       {\displaystyle
  \int\mathrm{d}\lambdaobs\,\lambdaobs\,g_{\lambda}(\lambdaobs)\,R(\lambdaobs)\,
    \int\mathrm{d}\lambdaemit\,\lambdaemit\,
    f_{\lambda}(\lambdaemit[1+z])\,Q(\lambdaemit)}
\right] \;\;\;,
\end{equation}
where, again, $R(\lambda)$ is defined to be the mean contribution to
the detector signal in the $R$ bandpass for a photon of wavelength
$\lambda$ and $Q(\lambda)$ is defined similarly.  Note that the
hypothetical standard source for the AB magnitude system, with
$g_{\nu}(\nu)$ constant, has
$g_{\lambda}(\lambda)\propto\lambda^{-2}$.

\section{Discussion}

To compute an accurate \kcorrection, one needs an accurate description
of the source flux density $f_{\nu}(\nu)$, the standard-source flux
density $g_{\nu}(\nu)$, and the bandpasses $R(\nu)$ and $Q(\nu)$.  In
most real astronomical situations, none of these is known to better
than a few percent, often much worse.  Sometimes, use of the AB system
seems reassuring (relative to, say, a Vega-relative system) because
$g_{\nu}(\nu)$ is known (ie, defined), but this is a false sense: In
fact the standard stars have been put on the AB system to the best
available accuracy; this involves absolute spectrophotometry of at
least some standard stars; this absolute flux information is rarely
known to better than a few percent.  The expected deviations of the
magnitudes given to the standard stars from a true AB system are
equivalent to uncertainties in $g_{\nu}(\nu)$.

The classical \kcorrection\ has $R(\nu)=Q(\nu)$.  This eliminates the
integrals over the standard-source flux density $g_{\nu}(\nu)$.
However, it requires good knowledge of the source flux density
$f_{\nu}(\nu)$ if the redshift is significant.  Many modern surveys
try to get $R(\nu)\sim Q([1+z]\nu)$ so as to weaken dependence on
$f_{\nu}(\nu)$, which can be complicated or unknown.  This requires
good knowledge of the absolute flux densities of the standard sources
if the redshift is significant.  This kind of absolute calibration is
often uncertain at the few-percent level or worse.

\bibliographystyle{hacked_apj}
\bibliography{apj-jour,ccpp}

\end{document}
