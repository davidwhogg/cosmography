% This is a LaTeX file
% by David W. Hogg

\documentclass[12pt,preprint]{aastex}

\begin{document}

\title{K-corrections}
\author{
  David W. Hogg\altaffilmark{1,2},
  Michael Blanton\altaffilmark{1},
  Daniel J. Eisenstein\altaffilmark{3},
  and Ivan K. Baldry\altaffilmark{4}\\
  \textsl{draft---2002 July 25}
}
\altaffiltext{1}{
  Center for Cosmology and Particle Physics,
  Department of Physics,
  New York University
}
\altaffiltext{2}{
  \texttt{david.hogg@nyu.edu}
}
\altaffiltext{3}{
  Steward Observatory,
  University of Arizona
}
\altaffiltext{4}{
  Department of Physics and Astronomy,
  The Johns Hopkins University
}

\begin{abstract}
The k-correction, used in relating the rest-frame absolute magnitude
in one broad photometric bandpass to the observed-frame apparent
magnitude in another broad bandpass, is described.
\end{abstract}

\section{Introduction}

The expansion of the universe provides astronomers with the great
benefit that recession velocities can be translated into radial
distances \citep[eg,][and references therein]{hogg99cosm}, but the
challenge that sources observed at different redshifts are observed,
by any particular instrument, at different rest-frame frequencies.
The transformations between observed and rest-frame broad-band
photometric measurements are known as ``k-corrections''
\citep{oke68a}.  Technically, the k-corrections described here include
a slight generalization from the original conception: we allow the
observed and rest-frame bandpasses to have arbitrarily different
shapes and position in frequency space.

\section{Equations}

Consider a source observed to have apparent magnitude $m_R$ when
observed through photometric bandpass $R$, for which we wish to know
its absolute magnitude $M_Q$ in emitted-frame bandpass $Q$.  In
classical work on k-corrections, $R$ and $Q$ would be identical, but
in the age of high-redshift observations, this is often not the case.
The k-correction $K_{QR}(z)$ for this source is defined by
\begin{equation}
\label{eq:definition}
m_R = M_Q + DM(z) + K_{QR}(z) \;\;\;,
\end{equation}
where $DM(z)$ is the distance modulus, defined by
\begin{equation}
DM(z) = 5\,\log_{10}\left[\frac{D_L^2}{(10~\mathrm{pc})^2}\right] \;\;\;,
\end{equation}
where $D_L$ is the luminosity distance \citep[eg,][]{hogg99cosm}.

The apparent magnitude $m_R$ of the source is related to its spectral
density of flux $f(\nu)$ (power per unit area per unit frequency) by
\begin{equation}
m_R = -2.5\,\log_{10}\left[
  \frac{\int\frac{\mathrm{d}\nu_o}{\nu_o}\,f(\nu_o)\,R(\nu_o)}
       {\int\frac{\mathrm{d}\nu_o}{\nu_o}\,g(\nu_o)\,R(\nu_o)}
\right] \;\;\;,
\end{equation}
where the integrals are over the observed frequencies $\nu_o$;
$g(\nu)$ is the spectral energy distribution of the ``standard
source'', which, for Vega-relative magnitudes, is Vega (or a weighted
sum of a certain set of A0 stars), and, for AB magnitudes, is a
theoretical constant source with $g(\nu_o)=3631~\mathrm{Jy}$ (where
$1~\mathrm{Jy}= 10^{-??}~\mathrm{W\,m^{-2}\,s^{-1}\,Hz^{-1}}$) at all
frequencies $\nu$; and $R(\nu)$ describes the bandpass, as follows:

The value of $R(\nu)$ at each freqency $\nu$ is the average
contribution of a photon of frequency $\nu$ to the output signal from
the detector.  If the detector is a photon counter, like a CCD, then
$R(\nu)$ is just the probability that a photon of frequency $\nu_o$
gets counted.  If the detector is a bolometer or calorimeter, then
$R(\nu)$ is just the fraction of photons of energy $\nu$ that get
absorbed into the detector, times the energy deposition $h\,\nu$ per
photon.  If $R(\nu)$ has been properly computed, there is no need to
write different integrals for photon counters and bolometers.

The absolute magnitude $M_Q$ is related to the spectral density of the
luminosity $L(\nu)$ (power per unit frequency) of the source by
\begin{equation}
m_R = -2.5\,\log_{10}\left[
  \frac{\int\frac{\mathrm{d}\nu_e}{\nu_e}\,L(\nu_e)\,             Q(\nu_e)}
       {\int\frac{\mathrm{d}\nu_e}{\nu_e}\,4\pi\,D_L^2\,g(\nu_e)\,Q(\nu_e)}
\right] \;\;\;,
\end{equation}
where the integrals are over emitted frequencies $\nu_e$, $D_L$ is the
luminosity distance, and $Q(\nu)$ is the equivalent of $R(\nu)$ but
for the bandpass $Q$.  As mentioned above, we are not requiring that
$Q=R$, so this is, technically, a generalization of the k-correction.
If the source is at redshift $z$, then
\begin{equation}
L(\nu_e) = \frac{4\pi\,D_L^2}{1+z}\,f(\nu_o) \;\;\;,
\end{equation}
where the factor of $(1+z)$ accounts for the fact that the flux is not
bolometric but flux density per unit freqency.  The factor would
appear in the numerator if we were using flux density per unit
wavelength.  There would be no factor if we were using flux density
per unit logarithm of frequency

Equation (\ref{eq:definition}) holds if the k-correction $K_{QR}(z)$
is
\begin{equation}
\label{eq:kcorrection}
K_{QR}(z) = -2.5\,\log_{10}\left[
  \frac{\int\frac{\mathrm{d}\nu_o}{\nu_o}\,f(\nu_o)\,R(\nu_o)\,
        \int\frac{\mathrm{d}\nu_e}{\nu_e}\,
          g(\nu_e)\,Q(\nu_e)}
       {\int\frac{\mathrm{d}\nu_o}{\nu_o}\,g(\nu_o)\,R(\nu_o)\,
        \int\frac{\mathrm{d}\nu_e}{\nu_e}\,\frac{1}{1+z}\,
          f\left(\frac{\nu_e}{1+z}\right)\,Q(\nu_e)}
\right] \;\;\;,
\end{equation}
where factors of $4\pi\,D_L^2$ have been cancelled out.  This
(\ref{eq:kcorrection}) can be taken to be an operational definition,
therefore, of the k-correction, from observations through bandpass $R$
of a source whose absolute magnitude $M_Q$ through bandpass $Q$ is
desired.

\section{Discussion}

To compute an accurate k-correction, one needs an accurate description
of the source flux density $f(\nu)$, the standard-source flux density
$g(\nu)$, and the filter curves $R(\nu)$ and $Q(\nu)$.  In most real
astronomical situations, none of these is known to better than a few
percent, often much worse.  Sometimes, use of the AB system seems
reassuring because $g(\nu)$ is known (ie, defined), but this is a
false sense, since in fact there is an uncertainty about the absolute
calibrations of the actual standard stars used, which is equivalent to
an uncertainty in $g(\nu)$.  Again, for many experiments, this
uncertainty is larger than a few percent.

The classical k-correction has $R(\nu)=Q(\nu)$.  This eliminates the
integrals over the standard-source flux density $g(\nu)$.  However, it
requires good knowledge of the source flux density $f(\nu)$ if the
redshift is significant.  Many modern surveys try to get $R(\nu)\sim
Q([1+z]\nu)$ so as to weaken dependence on $f(\nu)$, which can be
complicated or unknown.  This requires good knowledge of the standard
sources if the redshift is significant.

\bibliographystyle{../../nyu-astro/tex/apj}
\bibliography{../../nyu-astro/tex/apj-jour,../../nyu-astro/tex/ccpp}

\end{document}
